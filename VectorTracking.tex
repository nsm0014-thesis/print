\chapter{Proposed Navigation Filter Architecture}
Scalar tracking loops discussed in Chapter~3 are critically important to receivers as they adapt the replica signal to match the receiver signal data for proper decoding. However, these scalar loop filters assume a static noise bandwidth, regardless of receiver or satellite dynamics. If either platforms have unmodeled dynamics unknown before processing, these bandwidths can permit too much noise into the navigation processing solution or, do not allow enough of the signal to be processed, neglecting these dynamics. One solution to this problem is implementing an adaptive Kalman filter to optimally select bandwidths~\cite{}. In this case, the Kalman filter estimates the proper bandwidth based on discriminator residuals, but is agnostic to the dynamics of the receiver or the satellite dynamics. The addition of an adaptive Kalman filter is in an improvement, but leaves a lot to be desired as each channel is still be tracked individually, resulting in low-powered channels having a high likelihood of being lost.

Another, more optimal, solution is to estimate the receiver and satellite dynamics at every integration period. This requires an updated estimate of the navigation solution to be updated at every integration period, which is suitable as long as four satellites are transmitting to the receiver. This approach combines the adaptive bandwidth from~\cite{} along with knowledge of the receiver and satellite dynamics stemming from the navigation solution. This closed-loop approach is known as \textit{vector tracking} and will be discussed in greater detail later on in this chapter. Specifically, the Vector Delay and Frequency Lock Loop (VFDLL) is the vector tracking implementation used in this thesis.

To build on the attractive approach that vector tracking brings to processing received signal data, the novelty of this work proposes an addendum to the existing navigation filter architecture by adding a Flight Vehicle Dynamic Model (FVDM) to predict the trajectory of flight vehicle in time to better assist with the processing of the signal. Along wth the discussion of vector tracking, this chapter will describe the process model in the Extended Kalman Filter (EKF), and the measurement update that provides corrections to estimated state of the receiver in time.
\section{Vector Delay and Frequency Lock Loop}
Vector tracking first utilized a Vector Delay Lock Loop (VDLL) and was proposed by~\cite{}. In a VDLL, the EKF provides continuous estimates of the code frequency, updating the DLL, improving overall tracking performance. Later on,~\cite{}, explores tracking both code frequency and carrier frequency in an EKF, coined the VDFLL. This method showed great improvements over scalar tracking algorithms and moderate improvements over the VDLL. The VFDLL proves best when tracking weaker GNSS signals under high dynamic stress~\cite{}. Furthermore, recent analyses from~\cite{} prove the VFDLL has improved resilance to multipath delay as well. A block diagram of the VFDLL is shown in Figure~\ref{fig:VFDLL}.

\begin{figure}[!ht]
    \centering
    \includegraphics[width=0.45\linewidth]{Figures/VectorTracking.drawio.png}
    \caption{Block diagram of the VFDLL used in this work}\label{fig:VFDLL}
\end{figure}

From Figure~\ref{fig:VFDLL}, the \textit{RF Front End} block refers to the discussion in Section~3.1. The signal correlation block represent Equation~\ref{eq:correlators} and also the the FLL and DLL discriminators specified in Equations~\ref{eq:FLLdisc} and~\ref{eq:DLLdisc}, respectively. The basic flow of the VFDLL requires the receiver to know its position and the positions of the satellites \textit{a priori}. Preferably, initial positions and satellite positions that fed into the VFDLL are from processing the received signal for a length of time required to decode the navigation message using scalar tracking discussed in Chapter 3. The requirements of the EKF in Figure~\ref{fig:VFDLL} are residual pseudorange and pseudorange-rates in the form of outputs from the discriminators. The EKF uses the measurements from the current signal correlations to improve the position estimate of the receiver. Using the ephemeris of the satellites and the corrected position estimates, new code phase and carrier frequency estimates are generated for the next integration period. To improve the estimated position from the EKF, the FVDM is used as the process model to propagate the non-linear motion of the aircraft in time. The next section covers the time update and measurement update within the EKF\@.

\section{Deeply Coupled GPS and FVDM Navigation Filter}
As stated previously, the VFDLL replaces the scalar DLLs and FLLs with a single EKF\@. This sections describes the design of the EKF for the proposed navigation filter. The EKF for this work represents a position-state filter where the state vector is defined by Equation~\ref{eq:stateVector}.

\begin{equation}\label{eq:stateVector}
    \mathbf{X} =
    \begin{bmatrix}
        \mathbf{X}_V & \mathbf{X}_{\omega} & \mathbf{X}_P & \mathbf{X}_{\psi} & \mathbf{X}_t \\
    \end{bmatrix}^T
\end{equation}

The essential elements for the state vector are sectioned into five terms.~\(\mathbf{X}_V\) (Equation~\ref{eq:velVector}) describes the velocity states of the aircraft from Earth to body with respect to the Local Navigation frame.

\begin{equation}\label{eq:velVector}
    \mathbf{X}_V =
    \begin{bmatrix}
        V_N & V_E & V_D \\
    \end{bmatrix}
\end{equation}

The angular rates (\(\mathbf{X}_{\omega}\)) are represented from inertial to body and kept in the body frame (Equation~\ref{eq:omegaVector}).

\begin{equation}\label{eq:omegaVector}
    \mathbf{X}_{\omega} =
    \begin{bmatrix}
        p_{ib}^b & q_{ib}^b & r_{ib}^b \\
    \end{bmatrix}
\end{equation}

The position estimates of the aircraft are from Earth to body with respect to the Local Navigation frame (Equation~\ref{eq:posVector}), similar to the velocity states.

\begin{equation}\label{eq:posVector}
    \mathbf{X}_P =
    \begin{bmatrix}
        \lambda_{eb}^n & L_{eb}^n & h_{eb}^n \\
    \end{bmatrix}
\end{equation}

Equation~\ref{eq:eulVector} represents the Euler angles of the aircraft, represented from Earth to body.

\begin{equation}\label{eq:eulVector}
    \mathbf{X}_{\psi} =
    \begin{bmatrix}
        \phi_{bn} & \theta_{bn} & \psi_{bn} \\
    \end{bmatrix}
\end{equation}

Completing the state vector are the clock terms that represent estimates of the clock bias and clock drift of the receiver during flight (Equation~\ref{eq:clkVector}). The clock bias and clock drift are scaled by the speed of light (\(c\)) to give them units of meters and meters/second, respectively.

\begin{equation}\label{eq:clkVector}
    \mathbf{X}_t = \begin{bmatrix}
        c\delta t & c\delta\dot{t} \\
    \end{bmatrix}
\end{equation}

The dynamics of the aircraft are defined by

\begin{equation}\label{eq:eulerIntegration}
    \dot{\mathbf{X}} = F\left(\mathbf{X},f_{ib}^b,M_{ib}^b\right) + \mathbf{B}_{dyn}w_{dyn} + \mathbf{B}_{clk}w_{clk},
\end{equation}

where \(\mathbf{B}_{dyn}\) is the noise distribution matrix related to the dynamics (Equation~\ref{eq:Bdyn}).

\begin{equation}\label{eq:Bdyn}
    \mathbf{B}_{dyn} =\begin{bmatrix}
        \mathbf{I}_{6,6} & \mathbf{0}_{6,2} \\
        \mathbf{0}_{8,6} & \mathbf{0}_{8,2} \\
    \end{bmatrix}_{14,8}
\end{equation}

\(w_{dyn}\) is the disturbance vector for the aircraft dynamics, as shown in Equation~\ref{eq:wDyn}. Equation~\ref{eq:Bdyn} shows that only the linear and angular accelerations are affected by the disturbances, but these errors trickle down into the kinematic equations for the position and Euler derivatives.

\begin{equation}\label{eq:wDyn}
    w_{dyn} = \begin{bmatrix}
        \sigma^2_{V_N} & \sigma^2_{V_E} & \sigma^2_{V_E} & \sigma^2_{p} & \sigma^2_{q} & \sigma^2_{r}
    \end{bmatrix}^T
\end{equation}

\(\mathbf{B}_{clk}\) and \(w_{clk}\) are the noise distribution matrix (Equation~\ref{eq:Bclk}) and clock variance vector (Equation~\ref{eq:wClk}), respectively.

\begin{equation}\label{eq:Bclk}
    \mathbf{B}_{dyn} =\begin{bmatrix}
        \mathbf{0}_{6,6} & \mathbf{0}_{12,2} \\
        \mathbf{0}_{8,6} & -\mathbf{I}_{2,2} \\
    \end{bmatrix}_{14,8}
\end{equation}

\begin{equation}\label{eq:wClk}
    w_{clk} = \begin{bmatrix}
        \sigma^2_b & \sigma^2_d \\
    \end{bmatrix}^T
\end{equation}

\(\sigma^2_b\) and \(\sigma^2_d\) are rule of thumb variances based on the clock embedded into the receiver. For the receiver simulated in this work, an Oven Controlled Crystal Oscillator (OCXO) is used. More information on calculating the clock variance based on oscillator type can be found in~\cite{}.

\(F\) is a set of non-linear equations that define the motion of the aircraft as a function of the current state in time (\(\mathbf{X}\)), the forces in the body frame (\(f_{ib}^b\)) and the moments about the body frame (\(M_{ib}^b\)). Calculation of the forces and moments were discussed in Chapter 2.

Calculations of the forces and moments and moments must be done in the body frame. Rotating all of the equations that build the total forces and moments acting onto the airframe into a global or local navigation reference frame would induce more complexity than necessary. However, because of the measurements generated from the correlators and discriminators are composed in the ECEF reference frame, the FVDM must be propagated in a reference frame that respective the curvature of the Earth in a global frame. For this work, the equations of motion are rotated into the local navigation frame.

The state derivatives of the velocity components are defined in Equation~\ref{eq:acc}.

\begin{equation}\label{eq:acc}
    \begin{bmatrix}
        \dot{V_N} \\
        \dot{V_E} \\
        \dot{V_D} \\
    \end{bmatrix} =
    \mathbf{C}_{b}^{n}\frac{\mathbf{f}_{ib}^b}{m} - \left(2\mathbf{\Omega}_{ie}^n - \mathbf{\Omega}_{en}^n\right)
    \begin{bmatrix}
        V_N \\
        V_E \\
        V_D \\
    \end{bmatrix}
\end{equation}

The first term above represents the forces acting onto the airframe with respect to the body frame divided by the mass of aircraft, \(m\). For the purposes of this work, the mass of the aircraft is held constant. This specific force vector is rotated into the local navigation frame using \(\mathbf{C}_b^n\), defined in Equation~\ref{eq:ECEF2LNDCM}. The latter term in Equation~\ref{eq:acc} represents the rotation rate (Equation~\ref{eq:earthrotation}) of the Earth in skew-symettric form and the transport rate (Equation~\ref{eq:transportrate}) in skew-symettric form, both rotated into the local navigation frame and multiplied by the current velocity of the aircraft.

\begin{equation}\label{eq:earthrotation}
    \mathbf{\Omega}_{ie}^n =
    \omega_{ie}\begin{bmatrix}
        0                          & \sin\left(L_{eb}^n\right) & 0                          \\
        -\sin\left(L_{eb}^n\right) & 0                         & -\cos\left(L_{eb}^n\right) \\
        0                          & \cos\left(L_{eb}^n\right) & 0                          \\
    \end{bmatrix}
\end{equation}

Where \(\omega_{ie}\) is \(7.27\times10^{-5}\) radians/second.

\begin{equation}\label{eq:transportrate}
    \mathbf{\Omega}_{en}^n = \begin{bmatrix}
        0                & -\omega_{en,z}^n & -\omega_{en,y}^n \\
        \omega_{en,z}^n  & 0                & -\omega_{en,x}^n \\
        -\omega_{en,y}^n & \omega_{en,x}^n  & 0                \\
    \end{bmatrix}
\end{equation}

Where

\begin{equation}\label{eq:omega_en_n}
    \omega_{en}^n =
    \begin{bmatrix}
        V_E/(R_E + h_{eb}^n)               \\
        -V_N/(R_N + h_{eb}^n)              \\
        V_E\tan(L_{eb}^n)/(R_E + h_{eb}^n) \\
    \end{bmatrix}
\end{equation}

Above, \(R_E\) and \(R_N\) refer to the meridian and transverse radii of curvature as described in Equations~\ref{eq:meridiancurvature} and~\ref{eq:transversecurvature}, respectively.

The derivatives of the angular rates are defined as

\begin{equation}\label{eq:angacc}
    \begin{bmatrix}
        \dot{p}_{ib}^b \\
        \dot{q}_{ib}^b \\
        \dot{r}_{ib}^b \\
    \end{bmatrix} =
    {\mathbf{I}_{cg}^b}^{-1}\left[\mathbf{M}_{ib}^b -
        \begin{bmatrix}
            p \\
            q \\
            r \\
        \end{bmatrix} \times
        \left(\mathbf{I}_{cg}^b
        \begin{bmatrix}
            p \\
            q \\
            r \\
        \end{bmatrix}
        \right)
        \right],
\end{equation}
where \(\mathbf{I}_{cg}^b\) are the mass moments of inertia for the aircraft. For the purpose of this work, the aircraft is modelled symmetrically about each of the axes such that \(\mathbf{I}_{cg}^b\) only has terms along the diagonal.

The local navigation frame position derivatives are described by Equation~\ref{eq:posrate}.
\begin{equation}\label{eq:posrate}
    \begin{bmatrix}
        \dot{\lambda}_{eb}^n \\
        \dot{L}_{eb}^n       \\
        \dot{h}_{eb}^n       \\
    \end{bmatrix} =
    \begin{bmatrix}
        \frac{V_N}{R_N + h_{eb}^n}                                       \\
        \frac{V_E}{\left(R_E + h_{eb}^n\right)\cos\left(L_{eb}^n\right)} \\
        -V_D                                                             \\
    \end{bmatrix}\\
\end{equation}

The derivatives of the Euler angles are seen in Equation~\ref{eq:eulerRates}

\begin{equation}\label{eq:eulerRates}
    \begin{bmatrix}
        \dot{\phi}_{bn}   \\
        \dot{\theta}_{bn} \\
        \dot{\psi}_{bn}   \\
    \end{bmatrix} =
    \mathbf{C}_{\omega}
    \left(
    \begin{bmatrix}
            p \\
            q \\
            r \\
        \end{bmatrix} -
    \mathbf{C_n^b}\left(\omega_{ie}^n + \omega_{en}^n\right)
    \right)
\end{equation}

Calculation of the Euler rates is difference between the current angular rates of the aircraft and the rotation of the Earth along with the transport rate, similar to how the linear acceleration is calculated in Equation~\ref{eq:acc}. This difference is rotated by \(\mathbf{C}_{\omega}\) defined by Equation~\ref{eq:cOmega}.

\begin{equation}\label{eq:cOmega}
    \mathbf{C}_{\omega} =
    \begin{bmatrix}
        1 & \tan(\theta)\sin(\theta)    & \tan(\theta)\cos(\phi)      \\
        0 & \cos(\phi)                  & -\sin(\phi)                 \\
        0 & {\sin(\phi)}/{\cos(\theta)} & {\cos(\phi)}/{\cos(\theta)} \\
    \end{bmatrix}
\end{equation}

The last of the state derivatives are the clock terms. Both the clock drift and clock drift rate are scaled by the speed of light to give them units of \(\frac{m}{s}\) and \(\frac{m}{s^2}\).

\begin{equation}\label{eq:clkRates}
    \begin{bmatrix}
        c\delta \dot{t} \\
        c\delta\ddot{t} \\
    \end{bmatrix} =
    \begin{bmatrix}
        0 & 1 \\
        0 & 0 \\
    \end{bmatrix}
    \begin{bmatrix}
        c\delta {t}    \\
        c\delta\dot{t} \\
    \end{bmatrix}
\end{equation}

Once the state derivatives are calculated using the aforementioned equations of motions, they are integrated using Euler integrations to propagate the states forward in time. This provides the EKF with the predicted states for the current time step.

The other part of the prediction step in the EKF is form the predicted covariance matrix \(\mathbf{P}^-_{k}\). This is defined by Equation~\ref{eq:pminus}.

\begin{equation}\label{eq:pminus}
    \mathbf{P}^-_{k} = \mathbf{\Phi}\mathbf{P}^-_{k-1} \mathbf{\Phi}^T + \mathbf{Q_d}
\end{equation}

\( \mathbf{\Phi}\) is defined as the state transition matrix. The state transition matrix is composed of \(\mathbf{X}\) and the relationship with each state derivative \( \dot{\mathbf{X}}\). This done by taking the Jacobian and is represented by Equation~\ref{eq:stateJacobian}.

\begin{equation}\label{eq:stateJacobian}
    \mathbf{J} = \frac{\partial \dot{\mathbf{X}}_{14,1}}{\partial \mathbf{X}_{14,1}}
\end{equation}

The evaluated Jacobian \(\mathbf{J}\) is a square 14 row, 14 column matrix that varies as function of the forces and moments in time. These forces and moments can vary based on the disturbances and the controls inputs, so it must be evaluated at every time step. Because of the complexity of the Jacobian, it is solved using the symbolic toolbox in MATLAB.

The state propagation is continuous, but \(\mathbf{Phi}\) is discrete. This means that the Jacobian must be discretized. The discretization of the Jacobian from continuous to discrete is introduced by Equation~\ref{eq:Phi}

\begin{equation}\label{eq:Phi}
    \mathbf{\Phi} = \textrm{expm}(\mathbf{J}\Delta t)
\end{equation}

The discrete process noise covariance (\(\mathbf{Q}_d\)) encapsulates the disturbances onto the aircraft dynamics such that the EKF can better correct the states during the measurement update. To transform the process noise from continuous to discrete, Equation~\ref{eq:Qd} is used.

\begin{equation}\label{eq:Qd}
    \mathbf{Q_d} = \mathbf{\Phi}\mathbf{B}_w \left(ww^T\right) \mathbf{B}_w^T \mathbf{\Phi}^T \Delta t
\end{equation}

Above, \(\mathbf{B}_w\) is the noise distribution matrix that is augmented with \(\mathbf{B}_{dyn}\) and and \(\mathbf{B}_{clk}\).~\(w\) is the disturbance vector and is formed by concatenating the dynamic state disturbances and the clock term disturbances together (Equation~\ref{eq:w}). For the work presented in this thesis, the time step (\(\Delta t\)) is \( \frac{1}{200}\) seconds.

\begin{equation}\label{eq:w}
    w = \begin{bmatrix}
        w_{dyn} & w_{clk}
    \end{bmatrix}
\end{equation}

Once the state and covariance prediction are calculated, the \textit{a priori} part of the EKF is complete. The EKF will continue to accumulate prediction of the state and covariance until measurements from the vector tracking receiver are available. The next section covers the measurement update of the EKF, correcting the predicted covariance and predicted states.

\subsection{Update \textit{a posteriori}}
The measurement update in the EKF utilizes the current predicted covariance and measurement covariance to optimally correct the predicted states and predicted covariance. Equation~\ref{eq:L} shows the calculation of the Kalman gain.

\begin{equation}\label{eq:L}
    \mathbf{K}_k = \mathbf{P}^-_k \mathbf{H}^T_k\left(\mathbf{H}_k\mathbf{P}^-_k\mathbf{H}^T_k + \mathbf{R}_k\right)^{-1}
\end{equation}

The Kalman gain, \(\mathbf{K}\) is a function of the observation matrix, \(\mathbf{H}\), measurement covariance matrix, \(\mathbf{R}\), and the predicted covariance, described previously. The observation matrix maps the residual pseudorange and pseudorange rates to the local navigation states. This is demonstrated by Equation~\ref{eq:H}.

\begin{equation}\label{eq:H}
    \mathbf{H}_k = \begin{bmatrix}
        \mathbf{u}^{n,j}_{1,3} & \mathbf{0}_{1,3} & \mathbf{0}_{1,3}            & 0                & 0  & -1 \\
        \mathbf{0}_{1,3}       & \mathbf{0}_{1,3} & \mathbf{h}^j_{\rho p_{1,3}} & \mathbf{0}_{1,3} & -1 & 0  \\
    \end{bmatrix}_{2j,14}
\end{equation}

Using the predicted estimate of the receiver position Euler attitude, \(\mathbf{H}\) rotates the ECEF residuals to the local navigation frame. There are arrays of zeros where the measurements do not correlate to the states. The rotation of the unit vectors from ECEF to the local navigation frame is shown by Equation~\ref{eq:nav_u} and the rotation for the position measurements is given by Equation~\ref{eq:nav_h}.

\begin{equation}\label{eq:nav_u}
    \mathbf{u}^{n,j} = \mathbf{C}_e^n \mathbf{u}^{e,j}
\end{equation}

\begin{equation}\label{eq:nav_h}
    \mathbf{h}^j_{\rho p} =
    \begin{bmatrix}
        \left(R_N + h^n_{eb}\right)\mathbf{u}^{n,j}_N               \\
        \left(R_E + h_{eb}^n\right)\cos(L^n_{eb})\mathbf{u}^{n,j}_E \\
        -\mathbf{u}^{n,j}_D                                         \\
    \end{bmatrix}
\end{equation}

The measurement covariance matrix was described as the weighting measurement, \(\mathbf{W}\), in Chapter 3. The variances of the measurements are still calculated as a function of the carrier-to-noise ratio. Once the Kalman gain is calculated for the current time step, it can be used with the measurement vector, \(\Delta z\), to update the predicted state estimate (Equation~\ref{eq:deltaZ} and~\ref{eq:xplus}).

\begin{equation}\label{eq:deltaZ}
    \Delta z = \begin{bmatrix}
        \Delta\dot{\rho}^{\,1} & \Delta{\rho}^{\,1} & \hdots & \Delta\dot{\rho}^{\,j} & \Delta{\rho}^{\,j} \\
    \end{bmatrix}^T
\end{equation}

\begin{equation}\label{eq:xplus}
    \mathbf{X}_k^+ = \mathbf{X}^-_k + \mathbf{K}_k\Delta Z
\end{equation}

The last step in the measurement update is correcting the predicted covariance (Equation~\ref{eq:pplus}). The predicted covariance will drift based on the process noise covariance matrix and correcting the covariance by using the newly calculated Kalman gain will inform the filter of the confidence in the process model. 

\begin{equation}\label{eq:pplus}
    \mathbf{P}^+_k = \left(\mathbf{I} - \mathbf{K}_k\mathbf{H}_k\right)\mathbf{P}^-_k
\end{equation}

After the measurement update of the EKF is complete, the estimated states and covariance matrices redefine the predicted state and covariances, closing the loop in the proposed navigation filter.

\section{Conclusions}
Vector tracking is a closed loop solution to estimating navigation states in a GPS receiver. The proposed navigation filter augments the existing VFDLL architecture by instantiating the FVDM as the process model in the EKF\@. This chapter covered the differences between scalar and vector tracking loops and expanded on the FVDM is utilized to propagates the states of aircraft while also using correlator-level GPS measurements to correct the states and confidence in time. This chapter covered the additions to the VFDLL architecture that did not exist before. For a more nuanced implementation of the VFDLL, the reader is asked to read the sources cited throughout the chapter for more information.