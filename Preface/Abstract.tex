\documentclass[Thesis]{subfiles}
\begin{document}
\begin{abstract}
    % What is being presented in this work?

    % Why is it important?

    % How is the sensor fusion algorithm going to work?

    % What are the intricacies of the Flight Vehicle Dynamic Model?

    % Why use GPS measurements?

    % What results are shown in this thesis?
    
    As modern technology trends towards autonomous aerial and ground vehicles, the need for high-fidelity simulations is ever present. This thesis investigates the coupling of a high-fidelity Flight Vehicle Dynamic Model (FDVM) with simulated Global Positioning System (GPS) measurements in both healthy and degraded scenarios. The FDVM is developed to simulate real-world fixed wing aircraft and emulate measurements from an Inertial Measurement Unit (IMU) and other sensors these same aircraft may have equipped. The sensors simulated in this work suffer from noise, negatively affecting the measurements they provide. The adaptability of the FDVM allows for different aircraft to be analyzed, but this work focuses on the simulation of a Diamond DA-40, single propeller plane. GPS measurements are used to highlight realistic scenarios of degradation or multipath that may occur during important flight segments such as landing. While GPS can provide a true global position, the frequency of its measurements are lower when compared to the sensors simulated in this thesis. Sensor fusion algorithms are robust and well-documented; therefore, a closely coupled GPS and Inertial Navigation System is presented along with details of the high-fidelity FVDM.
\end{abstract}
\end{document}
