\chapter{\textbf{Scenario Implementation and Results}}
% Introduction
This chapter presents results of two trajectories using the proposed navigation filter and its underlying components discussed in previous chapters. Each trajectory is subject to varying degrees of signal interference across multiple Monte-Carlo simulations and will be discussed thoroughly to avoid confusion. First, the chapter starts with a discussion of the first trajectory and configuration file used for simulation. Following a description of trajectory one, Monte-Carlo analyses highlight the strong and weak points of the proposed navigation filter. Once the analyses of the first trajectory conclude, a description of trajectory two is presented, followed by a similar Monte-Carlo statistical analyses. Each Monte-Carlo case is composed of 100-run simulations for each trajectory subject to seven cases of signal degradation. To show the improved performance of the proposed navigation filter, a constant-velocity, kinematic model VDFLL~\cite{grierPositionNavigationTiming} is shown as a comparison. This standard VDFLL processes the same trajectories at the same cases of signal degradation.

\section{\textbf{First Trajectory}}
For the first trajectory, the aircraft is simulated for a straight flight path while maintaining a constant altitude of 500 meters above sea level (Figure~\ref{fig:trajectory1}).

\begin{figure}[!ht]
    \centering
    \includegraphics[width=\linewidth]{Figures/Results/trajectory1.png}
    \caption{(Left) Top-view of simulated flight path for trajectory one. (Right) Altitude of flight path for trajectory one.}\label{fig:trajectory1}
\end{figure}

It is assumed that the receiver knows it position beforehand when the simulation begins. For the first trajectory, the receiver aboard the aircraft is subject to seven different cases of interference that degrade the signals from the nine tracked channels (Table~\ref{tbl:interferenceCases}). The receiver is subject to these degraded power levels for the entirety of the 60 second simulations.

\begin{table}[!ht]
    \caption{Signal power for each case applied to each trajectory.}\label{tbl:interferenceCases}
    \centering
    \begin{tabular}{cc}
        \toprule
        Case & \(C/N_0\) [dB-Hz] \\
        \midrule
        1    & 45                \\
        2    & 35                \\
        3    & 25                \\
        4    & 22                \\
        5    & 20                \\
        6    & 18                \\
        7    & 16                \\
        \bottomrule
    \end{tabular}
\end{table}

From Table~\ref{tbl:trajectory}, several parameters are configurable to meet the desired simulation of the user.

\begin{table}[!ht]
    \caption{Initial conditions for simulated trajectory from Figure~\ref{fig:trajectory1}.}\label{tbl:trajectory}
    \centering
    \begin{tabular}{lcc}
        \toprule
        \textbf{Condition}       & \textbf{Value}                                      & \textbf{Units}                     \\
        \midrule
        Date                     & June 15, 2022                                       & \textit{DateTime} Object           \\
        Duration                 & 300                                                 & \(s\)                              \\
        Monte-Carlo Runs         & 100                                                 & {--}                               \\
        Frequency                & 200                                                 & Hz                                 \\
        Trajectory               & \textit{StraightFlightPath.mat}                     & See Chapter 5                      \\
        Velocity Disturbance     & \(\left[300, \; 300, \; 0\right]\)                  & \(m/s\)                            \\
        Angular Rate Disturbance & \(\left[10^{-12}, \; 10^{-12}, \; 10^{-12}\right]\) & \(rad/s\)                          \\
        Clock Type               & OCXO                                                & {--}                               \\
        Initial Velocity         & \(\left[75, \; 0, \; 0\right]\)                     & \(m/s\)                            \\
        Initial Angular Rate     & \(\left[0, \; 0, \; 0\right]\)                      & \(rad/s\)                          \\
        Initial Position         & \(\left[0.65617, \; -2.1376, \; 500\right]\)        & \(\left[rad, \; rad, \; m\right]\) \\
        Initial Attitude         & \(\left[0, \; 0, \; 0\right]\)                      & \(rad\)                            \\
        Initial Clock Terms      & \(\left[0, \; 0\right]\)                            & \(\left[m, \; m/s\right]\)         \\
        Channel \(C/N_0\)        & \(45,\,35,\,25,\,22,\,20,\,18,\,16\)                & dB-Hz                              \\

        \bottomrule
    \end{tabular}
\end{table}

Date is used to pull the specified Rinex file from~\cite{nollCrustalDynamicsData2010}. This Rinex file is then parsed for its ephemeris and used to propagate the satellites during the simulation. Figure~\ref{fig:skyplot} shows the available satellites at the first time step for June 15, 2022 used in this work.

\begin{figure}[!ht]
    \centering
    \includegraphics[width=0.8\linewidth]{Figures/Results/skyplot.png}
    \caption{Orange dots signify satellite locations at the start of the simulations given the date of broadcast ephemeris. Block dots signify satellites that are in-view but are discarded due to the 10 degree mask angle used. The green triangle represents the initial receiver position.}\label{fig:skyplot}
\end{figure}

Duration and Monte-Carlo Runs specify the length of each simulation in seconds and the number of simulations for each scenario and/or case. For the work presented in thesis, 100 simulation Monte-Carlo runs are used for analyses. This may not seems substantial, but based on~\cite{khaghaniAssessmentVDMbasedAutonomous2018,khaghaniAutonomousVehicleDynamic2016,mwenegohaModelbasedTightlyCoupled2020} and time to completion for each simulation, 100 simulations is enough to show the general trend for statistical purposes. The trajectory is specified as a -mat file. The creation of the trajectory file is detailed in Chapter 5. Trajectory one is meant to act as a baseline case where it is expected that the deeply-coupled FVDM and constant-velocity kinematic model both perform well.  As explained in chapter 5, disturbances are modeled onto the trajectory of the aircraft as the FVDM is not perfect. External conditions such as wind and various atmospheric conditions alter the behavior of the Diamond DA-40 slightly. These disturbances are defined as Velocity and Angular Rate in Table~\ref{tbl:trajectory}. Clock Type is the embedded receiver clock modeled during the simulation. More information on the different types of clocks can be found in Chapter 4. For the purposes of this thesis, both trajectories and all cases of signal interference will use the OCXO as the embedded receiver oscillator. The initial states of the aircraft are defined by the accompanying variables. Since our trajectory specifies a mostly-north flight path, the north velocity component is specified to be 75 meters per second, while the other components are zero. Also, the pitch of the aircraft is specified as 4 degrees up, this is so the aircraft generates lift at the first couple time steps during the simulations and does not enter a stall upon initialization. Lastly, the initial position of the aircraft is specified as the first waypoint location, for simplicity. The last configurable parameter is the initial Channel \(C/N_0\) of the available satellites.


\subsection{\textbf{Monte-Carlo Analyses}}
From Monte-Carlo results, several parameters can analyzed for the performance improvements of the proposed navigation filter over the standard VDFLL kinematic model. This section begins with an analysis of the signal-level results between the two filters and follows with an analysis of the state estimate performance for a range of signal powers. The appendix features an extended account of all cases run for this trajectory for readers interested.

For the range of signal interference presented in table~\ref{tbl:interferenceCases}, the Root Mean Square Errors (RMSE) of both the code phase and carrier frequency shows the improved performance by using the deeply-coupled FVDM in GPS-challenged environments (Figure~\ref{fig:codecarrierstraight}).

\begin{figure}[!ht]
    \begin{subfigure}{.45\textwidth}
        \centering
        \includegraphics[width=1\linewidth]{Figures/straight/codephaseRMSEstraight.png}
    \end{subfigure}%
    \begin{subfigure}{.45\textwidth}
        \centering
        \includegraphics[width=1\linewidth]{Figures/straight/carrFreqRMSEstraight.png}
    \end{subfigure}
    \caption{Code phase and carrier frequency RMSE as a function of signal power, specified in dB-Hz.}\label{fig:codecarrierstraight}
\end{figure}

During simulation where the signal was slightly degraded or benign (i.e.\ channel power greater than \(35\) dB-Hz) the standard velocity implementation of the VDFLL actually performs better on average than the proposed navigation filter. This is most likely due to the FVDM becoming over confident during simulation. This is more evident from Tables~\ref{tbl:straight35FVDM} and~\ref{tbl:straight35CV} where the position estimates from the constant velocity kinematic model are marginally better.

\begin{table}[!ht]
    \caption{RMSE, STD, and maximum error from 100-run Monte Carlo simulation when the receiver is subject to a degraded signal power level of \(35\) dB-Hz.}\label{tbl:straight35FVDM}
    \centering
    \begin{tabular}{ccccc}
        \toprule
                  & Position [m] & Speed [m/s] & Clock Bias [m] & Clock Drift [m/s] \\
        \midrule
        RMSE      & 0.23443      & 0.066059    & 0.013062       & 0.002834          \\
        STD       & 0.11305      & 0.030674    & 0.17509        & 0.0032795         \\
        Max Error & 0.72197      & 0.22588     & 0.05835        & 0.009             \\
        \bottomrule
    \end{tabular}
\end{table}

\begin{table}[!ht]
    \caption{RMSE, STD, and maximum error from 100-run Monte Carlo simulation when the receiver is subject to a degraded signal power level of \(35\) dB-Hz.}\label{tbl:straight35CV}
    \centering
    \begin{tabular}{ccccc}
        \toprule
                  & Position [m] & Speed [m/s] & Clock Bias [m] & Clock Drift [m/s] \\
        \midrule
        RMSE      & 0.093008     & 0.087402    & 0.013897       & 0.0032735         \\
        STD       & 0.048771     & 0.038728    & 0.015186       & 0.0040279         \\
        Max Error & 0.25751      & 0.26045     & 0.033083       & 0.009487          \\
        \bottomrule
    \end{tabular}
\end{table}

However, when the signal power is less than \(35\) dB-Hz, the deeply-coupled FVDM presents steady performance improvements as the interference grows stronger. For carrier frequency error, the deeply-coupled filter breaks down at roughly \(16\) dB-Hz, where the theoretical \(8.33\) dB-Hz standard deviation from~\cite{lashleyPerformanceAnalysisVector2009} is met. For the standard VDFLL implementation, this criteria is met at roughly \(20\) dB-Hz. The probability that the vector tracking loops are able to maintain channel lock for an entire simulation is shown in Figure~\ref{fig:trackingprobability1}.

\begin{figure}[!ht]
    \centering
    \includegraphics[width=0.5\linewidth]{Figures/straight/trackingprobstraight.png}
    \caption{The probability that each navigation filter is able to maintain channel lock throughout the simulation across different levels of signal interference.}\label{fig:trackingprobability1}
\end{figure}

Based on Figure~\ref{fig:trackingprobability1}, the proposed navigation filter shows \(~100\% \) tracking probability \(5\) dB-Hz greater than the standard constant-velocity kinematic model. As stated previously, one of the benefits in utilizing the FVDM within a sensor fusion framework is the acknowledgment of aircraft behavior given a set of control inputs. This allows the presented filter to rely less on the provided correlator measurements from GPS\@.

The pose estimates from the navigation filter reflect the performance of the vector tracking loops to maintain lock under different levels of signal degradation. Figure~\ref{fig:GEOPLOT1}.

\begin{figure}[!ht]
    \begin{subfigure}{.45\textwidth}
        \centering
        \includegraphics[width=1\linewidth]{Figures/straight/20/GEOPLOT.png}
    \end{subfigure}%
    \begin{subfigure}{.45\textwidth}
        \centering
        \includegraphics[width=1\linewidth]{Figures/straight/25/GEOPLOT.png}
    \end{subfigure}
    \caption{Average Latitude and Longitude of the FVDM and standard VDFLL implementation compared to the truth trajectory. The left figure is when both simulations had a signal power of \(20\) dB-Hz. The right figure is with a signal power of \(25\) dB-Hz.}\label{fig:GEOPLOT1}
\end{figure}

Even with the straight, baseline trajectory, the standard constant velocity filter begins to drift slightly when the GPS measurements are unreliable. This is partially due to the non-linear velocity at the beginning of the simulation. However, even though this is the case, the proposed navigation filter has no problem maintaining accurate tracking estimates. Based on Figure~\ref{fig:codecarrierstraight}, the deeply integrated FVDM maintains great position estimates with each channel's signal power down to \(16\) dB-Hz (Figure~\ref{fig:GEOPLOT2}).

\begin{figure}[!ht]
    \centering
    \includegraphics[width=0.5\linewidth]{Figures/straight/16/GEOPLOT.png}
    \caption{Average Latitude and Longitude of the FVDM and standard VDFLL implementation compared to the truth trajectory when subject to a degraded signal power of \(16\) dB-Hz.}\label{fig:GEOPLOT2}
\end{figure}

For the position estimates for the standard VDFLL implementation, the greatest error is shown in the downward direction (Figure~\ref{fig:Altitude1}).

\begin{figure}[!ht]
    \begin{subfigure}{.45\textwidth}
        \centering
        \includegraphics[width=1\linewidth]{Figures/straight/20/ALTITUDE.png}
    \end{subfigure}
    \begin{subfigure}{.45\textwidth}
        \centering
        \includegraphics[width=1\linewidth]{Figures/straight/25/ALTITUDE.png}
    \end{subfigure}
    \caption{Average altitude estimates of the FVDM and standard VDFLL implementation compared to the truth trajectory. The left figure is when both simulations had a signal power of \(20\) dB-Hz. The right figure is with a signal power of \(25\) dB-Hz.}\label{fig:Altitude1}
\end{figure}

This is simply because the lack of geometric diversity between the satellites sending the measurements. One way to solve this problem would be to include LEO satellites or signals of opportunity for improved altitude estimates from more diverse measurements.

The assumption that the acceleration of the aircraft is zero-mean is poor and this can be seen by the speed estimates in the constant-velocity kinematic model (Figure~\ref{fig:Speed1})

\begin{figure}[!ht]
    \begin{subfigure}{.45\textwidth}
        \centering
        \includegraphics[width=1\linewidth]{Figures/straight/20/SPEED.png}
    \end{subfigure}
    \begin{subfigure}{.45\textwidth}
        \centering
        \includegraphics[width=1\linewidth]{Figures/straight/25/SPEED.png}
    \end{subfigure}
    \caption{Average speed estimates of the FVDM and standard VDFLL implementation compared to the truth trajectory. The left figure is when both simulations had a signal power of \(20\) dB-Hz. The right figure is with a signal power of \(25\) dB-Hz.}\label{fig:Speed1}
\end{figure}

When the correlator measurements from the receiver become unreliable, the constant velocity filter has no choice but to rely on its own prediction of aircraft velocity. On the fundamental basis that the velocity is the integral of a zero-mean acceleration, it leads the standard implementation to a heavily biased velocity prediction, thus deviating from the true states.

Apart from the position and velocities, both filters also estimate the bias and drift of the embedded clock. As stated before, the clock used during the simulations is the OCXO\@. Figure~\ref{fig:Clocks1} presents the average clock bias estimates for each filter at both \(20\) and \(25\) dB-Hz signal power, while Figure~\ref{fig:Clocks2} presents the average clock drift estimates from each filter for the same interference cases.

\begin{figure}[!ht]
    \begin{subfigure}{.45\textwidth}
        \centering
        \includegraphics[width=1\linewidth]{Figures/straight/25/CLOCKBIAS.png}\
    \end{subfigure}
    \begin{subfigure}{.45\textwidth}
        \centering
        \includegraphics[width=1\linewidth]{Figures/straight/35/CLOCKBIAS.png}
    \end{subfigure}
    \caption{Average clock bias estimates of the FVDM and standard VDFLL implementation compared to the truth trajectory. The left figure is when both simulations had a signal power of \(20\) dB-Hz. The right figure is with a signal power of \(25\) dB-Hz.}\label{fig:Clocks1}
\end{figure}

\begin{figure}[!ht]
    \begin{subfigure}{.45\textwidth}
        \centering
        \includegraphics[width=1\linewidth]{Figures/straight/25/CLOCKDRIFT.png}\
    \end{subfigure}
    \begin{subfigure}{.45\textwidth}
        \centering
        \includegraphics[width=1\linewidth]{Figures/straight/35/CLOCKDRIFT.png}
    \end{subfigure}
    \caption{Average clock drift estimates of the FVDM and standard VDFLL implementation compared to the truth trajectory. The left figure is when both simulations had a signal power of \(20\) dB-Hz. The right figure is with a signal power of \(25\) dB-Hz.}\label{fig:Clocks2}
\end{figure}

The clock model utilized from~\cite{wangKalmanFilterBasedIntegrity} is the same used on filters, so similar performance should be expected. However, the deeply-coupled navigation filter still out performs the constant-velocity EKF in both cases. The errors in clock bias and clock drift are directly linked to position and velocity estimates seen before (Figures~\ref{fig:GEOPLOT1} and~\ref{fig:Speed1}). That is, the less error on the positional estimates means less likelihood of more error on the clock bias estimates. The same can be said for the velocity and clock drift estimates between the two filters.

As mentioned previously, the first trajectory was a baseline, litmus test where it is expected that both the proposed navigation filter and the constant velocity kinematic model would perform similarly. Regardless of the objective for the trajectory, in cases where the signal interference left the signal power to be less than \(25\) dB-Hz, the deeply-coupled FVDM begins to show improved performance. This is further illustrated by Tables~\ref{tbl:straight20FVDM} and~\ref{tbl:straight20CV} where the RMSE, STandard Deviation (STD), and maximum error found from the 100-run Monte Carlo analysis are shown for a subjected signal power of \(20\) dB-Hz.



\begin{table}[!ht]
    \caption{RMSE, STD, and maximum error from 100-run Monte Carlo simulation when the receiver is subject to a degraded signal power level of \(20\) dB-Hz.}\label{tbl:straight20FVDM}
    \centering
    \begin{tabular}{ccccc}
        \toprule
                  & Position [m] & Speed [m/s] & Clock Bias [m] & Clock Drift [m/s] \\
        \midrule
        RMSE      & 0.83733      & 0.27649     & 0.14072        & 0.0087105         \\
        STD       & 0.32434      & 0.13394     & 0.13147        & 0.0072989         \\
        Max Error & 1.8783       & 0.90375     & 0.48541        & 0.024451          \\
        \bottomrule
    \end{tabular}
\end{table}



\begin{table}[!ht]
    \caption{RMSE, STD, and maximum error from 100-run Monte Carlo simulation when the receiver is subject to a degraded signal power level of \(20\) dB-Hz.}\label{tbl:straight20CV}
    \centering
    \begin{tabular}{ccccc}
        \toprule
                  & Position [m] & Speed [m/s] & Clock Bias [m] & Clock Drift [m/s] \\
        \midrule
        RMSE      & 24.567       & 0.88208     & 0.17748        & 0.0088036         \\
        STD       & 10.287       & 0.34973     & 0.14068        & 0.005286          \\
        Max Error & 36.193       & 2.0585      & 0.50134        & 0.022316          \\
        \bottomrule
    \end{tabular}
\end{table}

\clearpage

\section{\textbf{Second Trajectory}}
For the second trajectory, the aircraft is simulated for a more dynamic flight pattern while commanded to climb to an altitude of 1150 meters above sea level (Figure~\ref{fig:trajectory2}). The dynamics induced by trajectory two show the effectiveness of the deeply-coupled FVDM to predict the behavior of the aircraft due to the addition angular rates and Euler attitude being estimated. It is assumed that the receiver knows it position beforehand when the simulation begins. Similarly to the first trajectory, the receiver aboard the aircraft is subject to seven different cases of interference that degrade the signals from the nine tracked channels (Table~\ref{tbl:interferenceCases}). The receiver is subject to these degraded power levels for the entirety of the five-minute simulation.

\begin{figure}[!ht]
    \centering
    \includegraphics[width=\linewidth]{Figures/Results/trajectory2.png}
    \caption{(Left) Top-view of simulated flight path for the second trajectory. (Right) Altitude of second flight path where aircraft is commanded to climb to 1150 meters and then maintain the altitude for the remainder of the simulation.}\label{fig:trajectory2}
\end{figure}

It should be noted that the same cases of interference from the first trajectory still apply to trajectory two (Table~\ref{tbl:interferenceCases}). Furthermore, the only change in configuration between the first and second trajectory is the specified flight path; for this dynamic trajectory, \textit{SCurveFlightPath.mat} will be used in place of the \textit{StraightFlightPath.mat} used previously. That being said, the satellites found in-view utilizing the broadcast ephemeris file for the specified date are the same satellites used during the following simulations (Figure~\ref{fig:skyplot}).

\clearpage
\subsection{\textbf{Monte-Carlo Analyses}}
From Monte-Carlo results, several parameters can analyzed for the performance improvements of the proposed navigation filter over the standard VDFLL kinematic model. This section begins with an analysis of the signal-level results between the two filters and follows with an analysis of the state estimate performance for a signal power of both \(45\) and \(22\) dB-Hz. The appendix features an extended account of all cases run for this trajectory for readers interested.

\begin{table}[!ht]
    \caption{RMSE, STD, and maximum error from 100-run Monte Carlo simulation when the receiver is subject to a degraded signal power level of \(35\) dB-Hz.}\label{tbl:dyn35FVDM}
    \centering
    \begin{tabular}{ccccc}
        \toprule
                  & Position [m] & Speed [m/s] & Clock Bias [m] & Clock Drift [m/s] \\
        \midrule
        RMSE      & 0.023097     & 0.068845    & 0.028182       & 0.0026012         \\
        STD       & 0.12219      & 0.03264     & 0.021298       & 0.0029319         \\
        Max Error & 0.74388      & 0.28942     & 0.073095       & 0.008722          \\
        \bottomrule
    \end{tabular}
\end{table}

\begin{table}[!ht]
    \caption{RMSE, STD, and maximum error from 100-run Monte Carlo simulation when the receiver is subject to a degraded signal power level of \(20\) dB-Hz.}\label{tbl:dyn20FVDM}
    \centering
    \begin{tabular}{ccccc}
        \toprule
                  & Position [m] & Speed [m/s] & Clock Bias [m] & Clock Drift [m/s] \\
        \midrule
        RMSE      & 0.64521      & 0.2673      & 0.1911         & 0.010044          \\
        STD       & 0.34121      & 0.12531     & 0.18278        & 0.0073514         \\
        Max Error & 1.5828       & 0.76501     & 0.5855         & 0.23049           \\
        \bottomrule
    \end{tabular}
\end{table}

\begin{table}[!ht]
    \caption{RMSE, STD, and maximum error from 100-run Monte Carlo simulation when the receiver is subject to a degraded signal power level of \(35\) dB-Hz.}\label{tbl:dyn35CV}
    \centering
    \begin{tabular}{ccccc}
        \toprule
                  & Position [m] & Speed [m/s] & Clock Bias [m] & Clock Drift [m/s] \\
        \midrule
        RMSE      & 0.097444     & 0.080918    & 0.030903       & 0.0033113         \\
        STD       & 0.031969     & 0.035597    & 0.02659        & 0.0039156         \\
        Max Error & 0.1905       & 0.23272     & 0.076362       & 0.00917           \\
        \bottomrule
    \end{tabular}
\end{table}

\begin{table}[!ht]
    \caption{RMSE, STD, and maximum error from 100-run Monte Carlo simulation when the receiver is subject to a degraded signal power level of \(20\) dB-Hz.}\label{tbl:dyn20CV}
    \centering
    \begin{tabular}{ccccc}
        \toprule
                  & Position [m] & Speed [m/s] & Clock Bias [m] & Clock Drift [m/s] \\
        \midrule
        RMSE      & 196.98       & 6.0576      & 0.20098        & 0.010995          \\
        STD       & 126.68       & 0.92248     & 0.19358        & 0.0077631         \\
        Max Error & 431.72       & 8.468       & 0.62891        & 0.024964          \\
        \bottomrule
    \end{tabular}
\end{table}
% Signal Performance Figures
\begin{figure}[!ht]
    \centering
    \includegraphics[width=\linewidth]{Figures/FigurePlaceholder.png}
    \caption{Code frequency and carrier frequency error as a function of \(C/N_0\) level for trajectory one.~\textcolor{red}{[Expected to Change]}}\label{fig:truecodefreqerrror2}
\end{figure}


\begin{figure}[!ht]
    \centering
    \includegraphics[width=\linewidth]{Figures/FigurePlaceholder.png}
    \caption{Probability of channel lock as a function of \(C/N_0\) level for trajectory one.~\textcolor{red}{[Expected to Change]}}\label{fig:trackingprobability2}
\end{figure}


% State Estimation Figures
\begin{figure}[!ht]
    \centering
    \includegraphics[width=\linewidth]{Figures/resultsv2/Slide5.PNG}
    \caption{(Left) Receiver clock drift estimates subject to a signal power of \(35\) dB-Hz. (Right) Receiver clock drift estimates subject to a signal power of \(20\) dB-Hz.}\label{fig:clockerror451}
\end{figure}

\begin{figure}[!ht]
    \centering
    \includegraphics[width=\linewidth]{Figures/resultsv2/Slide11.PNG}
    \caption{Comparison of altitude estimates from the deeply-coupled FVDM and the standard VDFLL\@. The signal power is \(35\) dB-Hz (Left) and \(20\) dB-Hz (Right).}\label{fig:fixthis4}
\end{figure}

\begin{figure}[!ht]
    \centering
    \includegraphics[width=\linewidth]{Figures/resultsv2/Slide6.PNG}
    \caption{(Left) Speed estimate during trajectory one for a signal power of \(35\) dB-Hz. (Right) Speed estimate during trajectory one for a signal power of \(20\) dB-Hz.}\label{fig:clockerror221}
\end{figure}

\begin{figure}[!ht]
    \centering
    \includegraphics[width=\linewidth]{Figures/resultsv2/Slide7.PNG}
    \caption{(Left) Receiver clock bias estimates subject to a signal power of \(35\) dB-Hz. (Right) Receiver clock bias estimates subject to a signal power of \(20\) dB-Hz.}\label{fig:fixthis}
\end{figure}

\begin{figure}[!ht]
    \centering
    \includegraphics[width=\linewidth]{Figures/resultsv2/Slide4.PNG}
    \caption{(Left) Receiver clock drift estimates subject to a signal power of \(35\) dB-Hz. (Right) Receiver clock drift estimates subject to a signal power of \(20\) dB-Hz.}\label{fig:velocityerror221}
\end{figure}

\begin{figure}[!ht]
    \centering
    \includegraphics[width=\linewidth]{Figures/resultsv2/Slide9.PNG}
    \caption{(Left) Angular rate estimates from the proposed navigation filter subject to a signal power of \(35\) dB-Hz. (Right) Angular rate estimates from the proposed navigation filter subject to a signal power of \(20\) dB-Hz.}\label{fig:fixthis2}
\end{figure}

\begin{figure}[!ht]
    \centering
    \includegraphics[width=\linewidth]{Figures/resultsv2/Slide10.PNG}
    \caption{(Left) Euler attitude estimates from the proposed navigation filter subject to a signal power of \(35\) dB-Hz. (Right) Euler attitude estimates from the proposed navigation filter subject to a signal power of \(20\) dB-Hz.}\label{fig:fixthis3}
\end{figure}

\clearpage

\section{\textbf{Conclusions}}

This chapter presented a detailed explanation of the two trajectories used for the performance analysis of the proposed navigation filter compared to standard, constant-velocity kinematic model VDFLL\@. Trajectory one is baseline case where the aircraft is commanded to fly straight and maintain a constant altitude for the entirety of the simulation. Trajectory two is a more dynamic trajectory with alternating, banking turns throughout the simulation on top of the aircraft being commanded to climb and maintain an altitude of 1150 meters. Each trajectory was subject to a range of interference starting from benign signal power to low signal power. Results from trajectory one show that when flying a standard, unexcited trajectory, the standard VDFLL and the proposed navigation filter are comparable. Results from trajectory two showcase the improved performance from the deeply-coupled FVDM because of the inability in the standard EKF to predict accelerations and angular rates. However, the results from trajectory two also show that the FVDM is not the sole-solution due to the unobservable angular states when excitement is lost in the simulated trajectory. External sensors such as a IMU or multi-antenna GPS system could improve the system observability for better estimates. Improvements to the proposed navigation filter are discussed in greater details in the next chapter.