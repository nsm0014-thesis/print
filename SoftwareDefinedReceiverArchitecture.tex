\chapter{GNSS Software Defined Receiver Architecture}
Since 1993, the Global Positioning System (GPS) has provided users with capable hardware to determine their global position within seconds and in recent developments, a centimeter-level position error. GPS can be explained in 3 components: the satellite vehicles in space, control and transmission of signals, and the receiver processing component.

GPS satellites have undergone multiple improvements and upgrades since the first satellites were launched in 1978. These versions of satellites are based on their \textit{block}. Currently, Block III is the most advanced satellite orbiting Earth today. Early versions (Block I) of GPS satellites were used mainly for development and did not transmit signals to the public. Lessons learned from the Block I satellites were fully integrated into the Block II GPS satellites, where GPS became fully operational in 1993. While there were many subtle differences between Block I and Block II satellites, the most important difference is that these new satellites broadcasted a signal on 2 frequencies, coined \textit{L1}, \textit{L2}, and \textit{L2c}, where \textit{L2c} is intended for civilian use. Modern day Block III satellites transmit their signal on the same frequencies that Block II satellites have, with the addition of \textit{L5}. Table provides the frequencies that signals are transmitted on.

\begin{table}[h!]\label{tbl:GPSfreq}
    \centering
    \begin{tabular}{lc}
        \toprule
        \textbf{Name} & \textbf{Frequency [Mhz]} \\
        \midrule
        \textit{L1}   & \(1575.42\)              \\
        \textit{L2}   & \(1227.60\)              \\
        \textit{L2c}  & \(1227.60\)              \\
        \textit{L5}   & \(1176.45\)              \\
        \bottomrule
    \end{tabular}
\end{table}
%% Background/Introduction

\section{Front End}
The signals received by an antenna must be down converted and digitized before the processing of the signal can take place. The \textit{Front End} of the receiver performs this conversion through a series of amplifiers, filters, and a Analog-to-Digital Converter (ADC). First, a signal is received by a Right-Hand Circularly Polarized (RHCP) antenna. The antenna can be passive, but for scenarios where long cables are used, a powered, active antenna may be necessary. Because of the low received signal power that GNSS constellations provide, the signal is amplified through a series of Low Noise Amplifiers (LNA) and Band Pass Filters (BPF). The LNA raise the power of the received GPS signal and the BPF act as a first-step in removing non-GPS signals from processing. The last stage is passing the continuous received signal through the ADC where the signal is converted to digitized samples at a frequency of the receiver-embedded oscillator. The oscillator is filtered using a phase lock loop, described later. Table~\ref{tbl:clocks} provides a description of the clock typically utilized on GNSS hardware receivers.

\begin{table}[!ht]\label{tbl:clocks}
    \caption{Typical clocks embedded on hardware GNSS receivers.}
    \centering
    \begin{tabular}{cccc}
        \toprule
        Name & Abbreviation & Application \\
        \midrule
        Oxidized Crystal Oscillator & OCXO & \\
        Oxidized Crystal Oscillator & OCXO & \\
        Oxidized Crystal Oscillator & OCXO & \\
        Oxidized Crystal Oscillator & OCXO & \\
        \bottomrule
    \end{tabular}
\end{table}

The purpose in this mixing process is to transform the signal into a more manageable intermediate frequency while still maintaining the same modulation and Doppler applied to the signal.  

\section{Acquisition}

\section{Tracking Loops}
\subsection{Code Phase Tracking}
\subsection{Carrier Phase Tracking}
\subsection{Carrier Frequency Tracking}

\section{Navigation Algorithms}
\subsection{GNSS Measurements}
\subsection{Open-Loop Architectures}
\subsection{Vector Tracking}

\section{Conclusions}