\documentclass[../chapter_2.tex]{subfiles}
\addbibresource{../../Thesis.bib}

\begin{document}
% Overview of Piston and Propeller Model
A common way for small general aviation aircraft to generate propulsive forces is through use of a propeller. The Diamond DA-40 used in this thesis has a single 5 blade propeller attached to a shaft that rotates according to the torque generated by the aircraft engine. The following section describes the piston engine model, its accompanying propeller, and the electric governor that indirectly controls the thrust power generated by the aforementioned components.

% Manifold Pressure
The engine inlet manifold is a set of vents that allow the ambient air to feed directly into the combustion chambers where the oxygen in the air is used in combination with the fuel to generate power. Based on the commanded throttle input, these manifold vents can be closed or open to let in more or less air, and in return the engine delivers a proportional amount of power to the shaft.

% Engine Air Flow Rate
The by product of calculating the manifold pressure is also calculating the air flow rate that being combusted with the fuel to generate power to the propeller shaft. Because the air flow rate is a by product of the commanded throttle, the engine fuel to air ratio is maintained by a commanded mixture input. The mixture defines the ratio between the fuel and air volume in the engine combustion chambers. When the mixture is \textit{lean}, the chamber has larger volume of air compared to fuel. The opposite can be said when the mixture is \textit{rich}. Since the equation that govern the combustion efficiency are based on sociometric principles, the ideal fuel to air ratio is \textit{ideal}, where there equal parts air and equal parts fuel in the combustion chamber.

% Engine Shaft Power and Torque
The displacement of the piston as a result of the combustion between fuel and air in the combustion chamber correlate to an act of torque on the propeller shaft that results in the spinning of the propeller. 

% Electric Governor Model
The governor that exists in ground and flight vehicles exists such that drastic changes in throttle do not result in extreme ramps of torque that could structurally damage engine components. It limits the rate of commanded throttle to be linear so that rotational acceleration of the shaft and propeller is safely increased or decreased.

% Propeller Dynamics
The purpose of an aircraft propeller(s) is to increase the velocity of the ambient air around them such that the lifting surfaces on the aircraft can generate lift and keep the aircraft in flight. There are 3 main components to focus on when designing and manufacturing propellers. These are 
\begin{itemize}
    \item[i.] Material(s)
    \item[ii.] Number of Blades
    \item[iii.] Blade Geometry.
\end{itemize} 
While the focus of this thesis is not on details in propeller design, it is important to show how the history and differences between each of these 3 items affect the efficiency and performance a propeller has in generating thrust power for the aircraft.

\end{document}