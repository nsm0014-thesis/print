\documentclass[../chapter_2.tex]{subfiles}
\addbibresource{../../Thesis.bib}

\begin{document}
In order to more accurately calculate a handful of dynamics models studied in this work, a model of Earth's atmosphere is needed to provided ambient temperature, pressure, density, speed of sound, and wind. This thesis uses the International Standard Atmosphere (ISA) model to approximate ambient temperature, ambient pressure, ambient density, and the speed of sound given a certain height above Mean Sea Level (MSL). Using an assumed linear distribution for temperature as a function of altitude, the ISA model assumes hydrostatic equilibrium as seen by Equation \ref{eq:2.1},

\begin{equation}
    \frac{dP}{dh} = -\rho\,g,
    \label{eq:2.1}
\end{equation}

where $\frac{dP}{dh}$ is the vertical pressure gradient as a factor of air density, $\rho$, and Earth's gravity, $g$. After integrating Equation \ref{eq:2.1}, the ISA model uses the ideal gas law (Equation \ref{eq:2.2}) 

\begin{equation}
    P = \rho\,R\textsubscript{air}\,T
    \label{eq:2.2}
\end{equation}

to solve for the ambient pressure $P$, and density, $\rho$. A complete form of the ISA model is seen in Equations \ref{eq:2.3} and \ref{eq:2.4}.

\begin{equation}
    P = P_0\,\exp\left({\frac{-g\,\Delta h}{R\textsubscript{air}\,T}}\right)
    \label{eq:2.3}
\end{equation}

\begin{equation}
    \rho = \rho_0\,\exp\left({\frac{-g\,\Delta h}{R\textsubscript{air}\,T}}\right)
    \label{eq:2.4}
\end{equation}

where $P_0$ and $\rho_0$ are atmospheric layer values for pressure and density, respectively; $R\textsubscript{air}$ is the specific gas constant for air and $\Delta h$ is difference between the current altitude of the flight vehicle and altitude of the current atmospheric layer. The speed of sound is a function of temperature and can be calculated using Equation \ref{eq:2.5}

\begin{equation}
    a = \sqrt{\gamma\,R\textsubscript{air}\,T},
    \label{eq:2.5}
\end{equation}

where $a$ is the speed of sound in meters per second and $\gamma$ is the ratio of specific heats. Figure \ref{fig:atmos} describes these atmospheric parameters from MSL to 85,000 meters above sea level.

Along with the aforementioned atmospheric parameters, wind is a vital modeling parameter for flight vehicles. For smaller SWAP flight vehicles, small gusts of wind can greatly affect their dynamics \cite{raymerAircraftDesignConceptual2018}. This thesis uses an updated version of the Horizontal Wind Model \cite{drobEmpiricalModelEarth2008,drobUpdateHorizontalWind2015} using collected satellite data from the Nation Oceanic and Atmospheric Administration (NOAA). The model accepts the current position of the flight vehicle to provide a predicted wind vector in the meridional and zonal reference frame. Implementation of the ISA model and calculation of winds can be found under \textit{Atmosphere.m} in \cite{millerNsm0014thesis}.



\end{document}